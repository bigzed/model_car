\documentclass[10pt,oneside,a4paper]{article}
\usepackage[left=2cm,right=2cm,top=2cm,bottom=1cm,includeheadfoot]{geometry}
\usepackage{ngerman}
\usepackage[utf8]{inputenc}
% \usepackage{amsfonts,amssymb,amsmath,cancel,graphicx,textcomp}
\usepackage{amsfonts,amssymb,amsmath,graphicx,textcomp}
\usepackage{float}
\usepackage{color,xcolor}
\usepackage{url}
\usepackage{hyperref}
\usepackage{listings}
\usepackage{tikz}
\usepackage{fancyhdr}
\usepackage{gensymb}
\usetikzlibrary{arrows,shapes,snakes,automata,backgrounds,petri,positioning}

\hypersetup{
    colorlinks,
    citecolor=black,
    filecolor=black,
    linkcolor=black,
    urlcolor=black
}

\pagestyle{fancy}
\fancyhf{}
\fancyhead[L]{crash override : \\Steve Dierker, Semjon Kerner, Artur Jeske}
\fancyhead[C]{"Ubungsblatt 07}
\fancyhead[R]{Seite \thepage}
\renewcommand{\headrulewidth}{0.5pt}

% lstlisting mit Zeilennummerierung und grauen Kommentaren, Zeilenumbruch, etc. pp.
\lstset{
  numbers=left, numberstyle=\tiny, numbersep=5pt,
  tabsize=2,
  breaklines=true, breakindent=0pt, postbreak=\mbox{$\rightarrow\ \ $},
  showstringspaces=false,
  extendedchars=false,
  basicstyle=\small\ttfamily,
  commentstyle=\color{black!40},
  stringstyle=\color{black!40!blue},
  keywordstyle=\color{black!40!green}
}

% Komma Abstände bei Tausendern/Dezimalzahlen ans dt. anpassen
\mathcode`,="013B
\setlength{\parindent}{0em}
\setlength{\parskip}{0.5em}

\begin{document}
    Our Code can be found in:
    \url{https://github.com/bigzed/model_car/blob/version-4.0/catkin_ws/src/assignment8_oval_circuit/src}
    \section{Velocity Controller}
        Since we could not find documentation that gave us reasonable
        information on how often a tick appeared, we measured how many ticks
        happend within a time, with a specific rpm and from our error we
        were able to calculat the tick.
        That gave us a tick every X seconds. \\
        We used 2 different velocitys:
        \begin{itemize}
            \item 3 km/s (Y ticks / s) - low
            \item 6 km/s (Z ticks / s) - high
        \end{itemize}

        We used those two speeds to measure and plot the rpm: \\
        \includegraphics[scale=0.7]{pictures/velocity_plot.png}
    \section{Controll the car around an oval circuit}
        For this exercise we used ransac on the left and the right half of
        the screen. Steering happend in dependece to the average slope of the
        lines. We adapated the velocity to be high when steering is withing a
        threshold $V ^\circ$, otherwise to be low.
\end{document}
