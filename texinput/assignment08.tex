\documentclass[10pt,oneside,a4paper]{article}
\usepackage[left=2cm,right=2cm,top=2cm,bottom=1cm,includeheadfoot]{geometry}
\usepackage{ngerman}
\usepackage[utf8]{inputenc}
% \usepackage{amsfonts,amssymb,amsmath,cancel,graphicx,textcomp}
\usepackage{amsfonts,amssymb,amsmath,graphicx,textcomp}
\usepackage{float}
\usepackage{color,xcolor}
\usepackage{url}
\usepackage{hyperref}
\usepackage{listings}
\usepackage{tikz}
\usepackage{fancyhdr}
\usepackage{gensymb}
\usetikzlibrary{arrows,shapes,snakes,automata,backgrounds,petri,positioning}

\hypersetup{
    colorlinks,
    citecolor=black,
    filecolor=black,
    linkcolor=black,
    urlcolor=black
}

\pagestyle{fancy}
\fancyhf{}
\fancyhead[L]{crash override : \\Steve Dierker, Semjon Kerner, Artur Jeske}
\fancyhead[C]{"Ubungsblatt 08}
\fancyhead[R]{Seite \thepage}
\renewcommand{\headrulewidth}{0.5pt}

% lstlisting mit Zeilennummerierung und grauen Kommentaren, Zeilenumbruch, etc. pp.
\lstset{
  numbers=left, numberstyle=\tiny, numbersep=5pt,
  tabsize=2,
  breaklines=true, breakindent=0pt, postbreak=\mbox{$\rightarrow\ \ $},
  showstringspaces=false,
  extendedchars=false,
  basicstyle=\small\ttfamily,
  commentstyle=\color{black!40},
  stringstyle=\color{black!40!blue},
  keywordstyle=\color{black!40!green}
}

% Komma Abstände bei Tausendern/Dezimalzahlen ans dt. anpassen
\mathcode`,="013B
\setlength{\parindent}{0em}
\setlength{\parskip}{0.5em}

\begin{document}
    Our Code can be found in:
    \url{https://github.com/bigzed/model_car/blob/version-4.0/catkin_ws/src/assignment8_oval_circuit/src}
    \section{Velocity Controller}
        Since we could not find documentation that gave us reasonable
        information on how often a tick appeared, we had to measure the rate of
        ticks happening within a given distance. For that we let the model car drive
        and stopped it at 150 cm, which gave us 256 ticks, leaving us with
        around 170 ticks per meter.\\
        We used 2 different velocitys:
        \begin{itemize}
            \item 0,8 m/s (136 ticks/s) - "{}low"{}
            \item 1,2 m/s (204 ticks/s) - "{}high"{}
        \end{itemize}

        We wanted to use both "{}low"{} and "{}high"{} speed to measure and plot
        the rpm, yet we were not able to implement a valid PID controller in the
        time we assumed for this exercise.
        The best controller we were able to write oscillated
		unreasonable strong and did not converge to the desired RPM.
        Since once again all batteries were drained we were unable to proceed
        working - in this case a rosbag would not have been of much help.
        Also we could neither plot the graphs on the car because matplotlib
        was not installed and the ROS wifi has no connection to the internet,
        nor could we execute the programm on our machine, since the wifi connection
        was to error-prone to accumulate good values.
%        \includegraphics[scale=0.7]{pictures/velocity_plot.png}
    \section{Controll the car around an oval circuit}
        For this exercise we wanted to use ransac on both the left and the right
        half of the screen, to find two graphs for both the left and right markings.
        Steering would have happend in dependece to the average slope of the graphs.\\
        Ransac would have needed the scikit-learn library, which again was
        neither installed nor installable.\\
        In correspondance with other groups we also learned, that ransac is
        probably a poor choice, often detecting graphs wrongly.
        Therefore we prepared to use our approach from the last exercise.
        In that, we search for the first white pixel close to the center
        and let the car steer towards it, while smoothing steering with an
        accumulating shift register.
        
        We would have adapted the velocity to be "{}high"{} when steering is within a
        threshold close to straight steering, otherwise to be "{}low"{}.
        Sadly we didn't reach this point of the exercise due to huge administrative
        overhead with the car, the batteries, the wifi and lacking documentation.
\end{document}
