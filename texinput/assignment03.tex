\documentclass[10pt,oneside,a4paper]{article}
\usepackage[left=2cm,right=2cm,top=2cm,bottom=1cm,includeheadfoot]{geometry}
\usepackage{ngerman}
\usepackage[utf8]{inputenc}
% \usepackage{amsfonts,amssymb,amsmath,cancel,graphicx,textcomp}
\usepackage{amsfonts,amssymb,amsmath,graphicx,textcomp}
\usepackage{float}
\usepackage{color,xcolor}
\usepackage{url}
\usepackage{hyperref}
\usepackage{listings}
\usepackage{tikz}
\usepackage{fancyhdr}
\usepackage{gensymb}
\usetikzlibrary{arrows,shapes,snakes,automata,backgrounds,petri,positioning}

\hypersetup{
    colorlinks,
    citecolor=black,
    filecolor=black,
    linkcolor=black,
    urlcolor=black
}

\pagestyle{fancy}
\fancyhf{}
\fancyhead[L]{Steve Dierker, Semjon Kerner, Artur Jeske}
\fancyhead[C]{"Ubungsblatt 03}
\fancyhead[R]{Seite \thepage}
\renewcommand{\headrulewidth}{0.5pt}

% lstlisting mit Zeilennummerierung und grauen Kommentaren, Zeilenumbruch, etc. pp.
\lstset{
  numbers=left, numberstyle=\tiny, numbersep=5pt,
  tabsize=2,
  breaklines=true, breakindent=0pt, postbreak=\mbox{$\rightarrow\ \ $},
  showstringspaces=false,
  extendedchars=false,
  basicstyle=\small\ttfamily,
  commentstyle=\color{black!40},
  stringstyle=\color{black!40!blue},
  keywordstyle=\color{black!40!green}
}

% Komma Abstände bei Tausendern/Dezimalzahlen ans dt. anpassen
\mathcode`,="013B
\setlength{\parindent}{0em}
\setlength{\parskip}{0.5em}

\hypersetup{ urlcolor=blue }

\begin{document}
  \section{Aufabe 1}
    \href{https://github.com/bigzed/model_car/blob/version-4.0/catkin_ws/src/simple_parking_maneuver/src/parking_maneuver.py}{Parking Maneuver unchanged} \\
    \href{https://github.com/bigzed/model_car/blob/version-4.0/catkin_ws/src/simple_drive_control/src/drive_control.py}{Drive Control with calibrated angles} \\
    In drive\_control.py we calibrated the angles of the simple drive control
    node from \\
    15: angle\_left = 30 to angle\_left = 50 \\
    17: angle\_right = 150 to angle\_right = 170 \\
    for the parking maneuver to work properly.

    \href{https://github.com/bigzed/model_car/blob/version-4.0/texinput/videos/movie.mp4}{The video can also be found in our github}
  \section{Aufgabe 2}
  $^{B}_{A} T = \begin{pmatrix}
      cos 90\degree & - sin 90\degree & 0 & t_{x} \\
        sin 90\degree & cos 90\degree & 0 & t_{y} \\
        0 & 0 & 1 & t_{z} \\
        0 & 0 & 0 & 1 \\
    \end{pmatrix} =
  \begin{pmatrix}
        0 & -1 & 0 & -1 \\
        1 & 0 & 0 & 4 \\
        0 & 0 & 1 & 5 \\
        0 & 0 & 0 & 1 \\
    \end{pmatrix}$
  \section{Aufgabe 3}
    \begin{itemize}
        \item
            $\epsilon = \begin{pmatrix}
                cos(\frac{-3\pi}{2})\\
                0 * sin(\frac{-3\pi}{2})\\
                0 * sin(\frac{-3\pi}{2})\\
                1 * sin(\frac{-3\pi}{2})\\
                \end{pmatrix} = 
                \begin{pmatrix}
                    cos(-270\degree)\\
                0 * sin(-270\degree)\\
                0 * sin(-270\degree)\\
                1 * sin(-270\degree)\\
                \end{pmatrix} = 
                \begin{pmatrix}
                0\\
                0\\
                0\\
                -1\\
                \end{pmatrix}$ \\
            $R = 
                \begin{bmatrix}
                    1 - 2       & 2 * (0 - 0) & 2 * (0 + 0)\\
                    2 * (0 + 0) & 1 - 0 - 2   & 2 * (0 - 0)\\
                    2 * (0 - 0) & 2 * (0 + 0) & 1 - 0 - 0\\
                \end{bmatrix} =
                \begin{bmatrix}
                    -1 &  0 & 0 \\
                     0 & -1 & 0 \\
                     0 &  0 & 1 \\
                \end{bmatrix}$ \\
            $R*v = 
                \begin{bmatrix}
                    -1 &  0 & 0 \\
                     0 & -1 & 0 \\
                     0 &  0 & 1 \\
                \end{bmatrix} *
                \begin{pmatrix}
                    2 \\
                    0 \\
                    0 \\
                \end{pmatrix} =
                \begin{pmatrix}
                    -2 \\
                    0  \\
                    0  \\
                \end{pmatrix}$\\
        \item
            $\epsilon = \begin{pmatrix}
                0,5\\
                -0,5\\
                -0,5\\
                0,5\\
                \end{pmatrix} \\
            R = 
                \begin{bmatrix}
                    1 - 0,5 - 0,5 & 2 * (0,25 - 0,25) & 2 * (- 0,25 - 0,25)\\
                    2 * (0,25 + 0,25) & 1 - 0,5 - 0,5   & 2 * (-0,25 + 0,25)\\
                    2 * (-0,25 + 0,25) & 2 * (-0,25 - 0,25) & 1 - 0,5 - 0,5\\
                \end{bmatrix} =
                \begin{bmatrix}
                    0 & 0 & -1 \\
                    1 & 0 & 0 \\
                    0 & -1 & 0 \\
                \end{bmatrix}\\
            X = \begin{pmatrix}
                0\\
                1\\
                0\\
            \end{pmatrix}
            Y = \begin{pmatrix}
                0\\
                0\\
                -1\\
            \end{pmatrix}
            Z = \begin{pmatrix}
                -1\\
                0\\
                0\\
            \end{pmatrix}\\
            \epsilon_{0} = cos(\frac{\Theta}{2}) \rightarrow \Theta = 120\degree$
    \end{itemize}
  \section{Aufgabe 4}
  Since z must be orthogonal to y and x the vector z = (0, 0, sqrt(0.5)).
\end{document}
