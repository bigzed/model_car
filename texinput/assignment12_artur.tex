\documentclass[10pt,oneside,a4paper]{article}
\usepackage[left=2cm,right=2cm,top=2cm,bottom=1cm,includeheadfoot]{geometry}
\usepackage{ngerman}
\usepackage[utf8]{inputenc}
% \usepackage{amsfonts,amssymb,amsmath,cancel,graphicx,textcomp}
\usepackage{amsfonts,amssymb,amsmath,graphicx,textcomp}
% \usepackage{float}
\usepackage{color}
\usepackage{url}
\usepackage{hyperref}
\usepackage{listings}
\usepackage{tikz}
\usepackage{fancyhdr}
\usepackage{gensymb}
\usepackage[section]{placeins}
\usepackage{minted}
\usetikzlibrary{arrows,shapes,snakes,automata,backgrounds,petri,positioning}

\hypersetup{
    colorlinks,
    citecolor=black,
    filecolor=black,
    linkcolor=black,
    urlcolor=black
}

\pagestyle{fancy}
\fancyhf{}
\fancyhead[L]{crash override : \\Steve Dierker, Semjon Kerner, Artur Jeske}
\fancyhead[C]{"Ubungsblatt 12}
\fancyhead[R]{Seite \thepage}
\renewcommand{\headrulewidth}{0.5pt}

% lstlisting mit Zeilennummerierung und grauen Kommentaren, Zeilenumbruch, etc. pp.
\lstset{
  numbers=left, numberstyle=\tiny, numbersep=5pt,
  tabsize=2,
  breaklines=true, breakindent=0pt, postbreak=\mbox{$\rightarrow\ \ $},
  showstringspaces=false,
  extendedchars=false,
  basicstyle=\small\ttfamily,
  commentstyle=\color{black!40},
  stringstyle=\color{black!40!blue},
  keywordstyle=\color{black!40!green}
}

% Komma Abstände bei Tausendern/Dezimalzahlen ans dt. anpassen
\mathcode`,="013B
\setlength{\parindent}{0em}
\setlength{\parskip}{0.5em}

\begin{document}
\section{Time and Precision (10 Punkte)}
    \begin{itemize}
      \item Quellcode:
        \url{https://github.com/bigzed/model_car/blob/version-4.0/texinput/src/localization_assignment12.py}
      \item ohne Hindernisse:
        \begin{itemize}
          \item Zeit: unter $ 7s $
          \item Video:
          \url{https://raw.githubusercontent.com/bigzed/model_car/version-4.0/texinput/videos/u12_1.mp4}
        \end{itemize}
      \item mit Hindernisse:
        \begin{itemize}
          \item Zeit: ca. $ 11s $
          \item Video:
            \url{https://raw.githubusercontent.com/bigzed/model_car/version-4.0/texinput/videos/u12_2.mp4}
        \end{itemize}
    \end{itemize}

    \subsection{Herangehensweise}
      Wir haben zwar volle Punktzahl auf dem 11. Zettel bekommen, allerdings weist unser Programm
      nat\"urlich noch Fehler auf. Daher haben wir uns f\"ur diesen Zettel entschieden keine
      weiteren Sensoren zu benutzen, sondern weiter auf das \emph{fake GPS} f\"ur die Steuerung und
      den LIDAR f\"ur die Hinderniserkennung zu verlassen.

      \subsubsection{Steuerung}
        W\"ahrend des 11. Zettels sind uns folgende Probleme zur Steuerung aufgefallen:
        \begin{itemize}
          \item sehr sp\"ate Entscheidungen bzw.\ manchmal auch keine. Wir vermuteten Packetverluste
          \item zu schwacher Lenkwinkel
          \item falsche Entscheidungen in der oberen linken Kurve
          \item keine M\"oglichkeit w\"ahrend des Fahrens Geschwindigkeit und Look-Ahead zu
            konfigurieren
        \end{itemize}
        Um diese Probleme zu lösen haben wir uns folgendes überlegt:
        \begin{itemize}
            \item Als erstes haben wir den Kreismittelpunkt des oberen Kreises von $(215|196)$ auf $(215|200)$
            verschoben, da anscheinend die reale Welt nicht genau mit der Karte \"ubereinstimmt.
            \item Als n\"achstes haben wir die Lenk-Entscheidungen im oberen Kreissektor \"uberpr\"uft und uns ist aufgefallen, dass unsere Berechnung des Winkels zwischen Auto-Vektor und Ziel-Vektor nicht
            in jedem Sektor korrekt sind. Manchmal erhielten wir nicht den kleineren, sondern den
            gr\"oseren Winkel zwischen den beiden Vektoren.
            \item Als n\"achstes haben wir den Ausschlag des Lenkwinkels erh\"oht, sodass dieser seinen
            maximalen Wert schon bei einem Winkel von Ziel zu Auto von $90^\circ$ erreicht.
            \item Nach der Korrektur dieser Punkte konnten wir 400 Umdrehungen als Geschwindigkeit
            f\"ur den Motor einstellen und sicher fahren
            \item Nun haben wir die Hinderniserkennung von der Steuerung entkoppelt und sie beide in eigene
            Callbacks aufgeteilt, somit stehen die Lenkentscheidungen schneller zur Verf\"ugung.
            \item Zusammen mit einer GUI die wir erstellt haben um die Parameter \emph{desired\_speed},
            \emph{look\_ahead\_curve} und \emph{look\_ahead\_straight} w\"ahrend des Fahrens zu
            Ver\"andern konnten wir die optimale Konfiguration ermittlen.
            \item Bei einer Geschwindigkeit von $1000$ Umdrehungen, brauchen wir einen \emph{look\_ahead} von
            $120cm$ um das Resultat aus Video 1 zu erreichen. Bei der Kalibrierung ist uns aufgefallen,
            das der \emph{look\_ahead} der selbe f\"ur Kurven und Geraden seien sollte.
        \end{itemize}
        
      \subsubsection{Hinderniserkennung}
        W\"ahrend des 11. Zettels sind uns folgende Probleme zur Hinderniserkennung aufgefallen:
        \begin{itemize}
          \item LIDAR Position zu QR-Code Position zu Vorderachse nicht kalibriert
          \item trotz Erkennung kam es zu Kollisionen, da das Ausweichman\"over oder die Bremsung zu sp\"at bzw. zu schwach ausgef\"uhrt wurde
        \end{itemize}
        Um diese Probleme zu lösen haben wir uns folgendes überlegt:
        \begin{itemize}
            \item Als erstes haben wir die Position zueinander kalibriert und die Callbacks wie im letzten
            Abschnitt beschrieben entkoppelt. Das hat die Genauigkeit erh\"oht, allerdings sind die
            Entscheidungen teilweise immer noch zu sp\"at.
            \item Deswegen haben wir uns entschieden, sobald wir ein Hindernis erkennen und ein Ausweichman\"over einleiten, die Geschwindigkeit zu verringern und die Lenkung um den Faktor 2 zu
            verst\"arken. Das hat dazu gef\"uhrt das trotz sp\"ater Erkennung eine Kollision verhindert
            werden kann. Es bek\"ampft allerdings nur das Symptom nicht den Grund.
            Wir haben dies in der \mintinline{python}{ def get_lane(self, car_x, car_y): } bei der Abfrage 
            \mintinline{python}{if self.lane_is_free(obstacles, self.lane_id):} eingebaut.
            \item Der Grund f\"ur die sp\"ate Entscheidungen sind die Timestamps der \emph{fake GPS}
            Datenpakete. Um die LIDAR-Daten in das \emph{fake GPS} Koordinatensystem umzurechnen,
            brauchen wir eine GPS Information zu fast der selben Zeit, allerdings laufen auf den Autos
            und dem \emph{fake GPS} anscheinend keine \emph{NTP} Clienten. Die Fluktuation der
            Timestamps zum Empfangen auf dem Auto geht von $- 3s \text{ bis } + 3s$.
            \item Dieses Problem ist allein auf dem Auto nicht l\"osbar, allerdings k\"onnen wir die
            Auswirkungen verringern.
            \item Als erstes ersetzen wir den Timestamp des Pakets mit dem Zeitpunkt des Empfangens und
            eliminieren die Fluktuation, nehmen dabei aber in Kauf das LIDAR und GPS falsch ineinander
            \"uberf\"uhrt werden.
            \item Als n\"achstes warten wir f\"ur jedes LIDAR-Paket bis zu $150ms$ ob ein passendes GPS-Paket
            empfangen wird, falls nicht brechen wir ab und schreiben eine Nachricht auf die Konsole.
            \item Wir konnten so zwar die Pakete synchronisieren, allerdings warten wir schon bis zu $150ms$
            bis wir mit der Berechnung anfangen, somit sind die Ausweichman\"over meist noch zu sp\"at.
            \item Die n\"achsten Kurs-Teilnehmer w\"urden sich bestimmt freuen wenn die Zeit auf Autos und
            \emph{fake GPS} durch einen NTP-Server synchronisiert wird.
        \end{itemize}
\end{document}
